%\documentstyle[preprint,aps]{revtex}
\documentclass[prb,showpacs,twocolumn,aps,a4paper]{revtex4}
\usepackage{graphicx}
\usepackage{amsfonts}

\begin{document}

\title{Clausius-Mosotti in 2D}
\author{ }
\address{}
\date{\today}


\begin{abstract}
\noindent

\end{abstract}
\pacs{71.10.-w, 71.10.Fd, 71.27.+a, 78.20.Bh}
\maketitle

%\narrowtext

\section{}

We want to compute the dielectric constant of a system of polarizable
``molecules'' in 2D. For permanent dipoles at finite temperature we can assume a
Curie law for the polarizability $\alpha \sim/(k_B T)$. 

The particles are assuming to be distributed randomly in space but the
same result is valid for a square lattice. We apply a uniform field in
the y direction. Then all the particles acquire a dipole moment  
$p=\alpha E_{loc}$. The local field, $E_{loc}=E+E_d$, acting on one
molecule includes a contribution
from the external field $E$ and the dipole dipole interactions $E_d$
with all the other molecules. 

Now we compute $E_d$. 
The dipole field on a particle at the origin is given by: 
$$
E_d=\sum_{x,y} e_d(x,y) 
$$
with the field at the origin due to the dipole at position $(x,y)$
given by:
$$
e_d(x,y)= - p\left(\frac{1}{R^2}-2  \frac{ y^2}{R^4}\right)
$$
which can be obtain from Goetz expression for the energy of two
dipoles in 2D specialized in the present case and making the
derivative respect to one of the dipoles. 

Now the crucial point is that $E_d$ can be separated into two
contributions one from the dipoles inside a large circle and the other
from the dipoles outside. The contribution from inside the circle
cancels.  Indeed consider the contribution from the dipole at $(x,y)$
and the dipole at $(y,x)$ (in 3D one considers the cyclic permutations
of $(x,y,z)$):

$$e_d(x,y)+e_d(y,x)=- p\left(2 \frac{1}{R^2}-2  \frac{ y^2+x^2}{R^4}\right)=0 $$

Taking the radius of the circle $r$ very large we can compute the
contribution from the dipoles outside the circle using macroscopic
arguments.  The macroscopic polarization is $P=pn$ with $n$ the
density of molecules. The deferential of charge  on the border of the
circle at position $r (sin\theta, \cos\theta )$ (here $\theta$ is the
angle with the $y$-axis) is: 
$$
dq=- P \cos\theta r d\theta
$$
The electric field on the $y$ direction due to this charge is 
$$
dE=-\frac{dq \cos\theta}r
$$
and the total field in $y$ is 
$$
E_d=\int_0^{2\pi} d\theta P\cos^2\theta=\pi P
$$
The local field is $E_{loc}=E+\pi P$. Then $P=np=n\alpha (E+\pi P)$
and $P=\chi E$
with 
$$\chi=\frac {n\alpha }{1-\pi n \alpha},$$
which is the 2D generalization of Clausius-Mosotti.
We see that when $n \alpha(k_B T)>1/\pi$ the system becomes unstable and
becomes a ferroelectric therefore the ground state is ferroelectric,
actually a nematic since there is orientational order but not
positional order.  

Notice that the computation is very delicate. A cutoff in the
dipole-dipole interaction can spoil the result since will cancel
$E_d$. 

\bibliographystyle{prsty}
\bibliography{lorenzana}

\end{document}






